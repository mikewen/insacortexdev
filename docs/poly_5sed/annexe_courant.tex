\section{Mod�le de la boucle de courant}
\label{AnnexeCourant}


  La figure~\ref{FigAutom} donne le mod�le lin�aire de la
  boucle de courant � partir du sch�ma technique de la
  figure~\ref{FigRameEtMicro}

\begin{figure*}[h!]
  \centering
  \resizebox{0.7\linewidth}{!}{\inputxfig{./figures/bloc_autom}}
  \caption{Mod�le de la boucle de courant}
  \label{FigAutom}
\end{figure*}


  Le C167 commande le courant de consigne $V_{Ic}$ de la
  carte courant en agissant sur le rapport cyclique $\alpha$
  d'une sortie PWM. Pour cela la commande {\tt Fixe\_Rapport(float)} prend un argument de type {\tt float},
  not� $\sigma \in \left[ -1, +1 \right]$. Lorsque $\sigma$
  varie entre -1 et 1, le rapport cylique $\alpha$ varie
  entre 0 et 1 et la tension $V_{ic}$ entre +5V et -5V. On
  obtient donc directement le gain
  $K_{PWM}=\frac{V_Ic{}}{\sigma}=5$.

  La consigne de courant est compar�e � la mesure du courant
  d'induit donn�e par le capteur � effet Hall avec une
  sensibilit� $S_I$ de $0,550\,\, (V/A)$. La tension de consigne
  �tant born�e � $\pm{}K_{PWM} (V)$ on limite ainsi le
  courant d'induit dans la gamme
  $\pm{}\frac{K_{PWM}}{S_I}\approx\pm{}10\,\, (A)$.

  Le correcteur d'erreur est un proportionnel int�gral de la
  forme $C(p)=K_{PI}\frac{1+\tau_1\,p}{\tau_2\,p}$ qui
  permet donc d'annuler l'erreur statique. Sa sortie est une
  tension ($\pm8V$) qui pilote le rapport cyclique $\alpha$
  de la commande $In+$ du pont en H entre $\frac{1}{2} -
  \frac{1}{2}$ et $\frac{1}{2} + \frac{1}{2}$. Une variation
  du rapport cyclique de 1 est donc provoqu�e par une
  variation de la sortie de 11V, cela donne un gain
  $K_{MLI}=\frac{1}{11}$ ($In-$ est toujours en opposition
  de $In+$).



  Le pont en H command� par $In\pm$ permet d'imposer une
  tension de $\pm{}24V$ aux bornes du moteur. La moyenne
  $\bar{U}$ de cette tension varie entre -24V et +24V lors
  que $\alpha$ varie entre $-\frac{1}{2}$ et $\frac{1}{2}$
  autour de $\frac{1}{2}$. Le gain apport� par le pont en H
  est donc $K_H=\frac{24}{0,5}=48$.


  La fonction de transfert de l'induit du moteur
  $G_{elec}(p)$ est de la forme~:

\begin{eqnarray}
  G_{\mathit{elec}}(p)= &\frac{\bar{I}(p)}{\bar{U}(p)}=\frac{\frac{1}{R}}{1+\frac{L}{R}p}=\frac{K_{\mathit{elec}}}{1+\tau
    _{\mathit{elec}}.p}\\
  \mbox{avec} & K_{elec}=0,89 (A.V^{-1})\nonumber\\
  \mbox{et} &\tau_{elec}=2 (ms) \nonumber  
  \label{eq:Gelec}
\end{eqnarray}

On simplifie le p�le $\tau_{elec}$ avec le z�ro $\tau_1$ du
correcteur PI. On obtient ainsi une fonction de transfert en
boucle ouverte de forme int�grale pure. La boucle ferm�e est
donc un premier ordre $BF_{elec}=\frac{K_I}{1+\tau_I\,p}$
dont le gain statique d�pend uniquement de la sensibilit� du
capteur $K_I=\frac{1}{S_I}$. La constante de temps $\tau_I$
est fix�e directement par le gain du correcteur � 300 Hz
afin de bien att�nuer les oscillations du courant provoqu�e
par la PWM du hacheur (30 KHz).

Cette constante de temps �tant tr�s rapide devant la
dynamique m�canique d'une voiture de m�tro, on n�gligera son
influence.

\important{On peut ainsi mod�liser la boucle de courant
par le simple gain
$\frac{\bar{I}(p)}{\sigma(p)}=\frac{K_{PWM}}{S_I}\approx 10$,
lorsque la bande passante de $\sigma(p)$ n'exc�de pas 300Hz }    
