\section{Librairie de p�riph�riques}
\label{SSecGestionnairePeriph}
\label{PartPeriphs}


Une librairie vous est livr�e \og clef en main \fg{}
\texttt{driveurs\_2008a.c} et \texttt{driveurs\_2008a.h} qui permet de
g�rer ces p�riph�riques\footnote{Merci � Thierry Rocacher
  pour ces biblioth�ques de p�riph�riques}. Le fichier
d'ent�te contient toutes les informations quand �
l'utilisation de cette librairie dont voici les fonctions
principales.

\subsection{Acquisition de la position -- TIMER 3 --}
Le timer 3 en mode \textit{incremental interface mode} permet de mesurer la position angulaire relative d'une roues.

Les fonctions associ�es sont~:
\begin{description}
\item[void Init\_Position(int Position)] initialise la
  position relative mesur�e en nombre de pas (largeur d'un
  strie du codeur incr�mental de 4,254 mm environ ;-)~;
\item[int Lire\_Position(void)] donne la position
  relative en nombre de pas.
\end{description}

\subsection{Acquisition de la vitesse -- ADC --}
L'unit� charg�e d'acqu�rir la vitesse consiste � lancer la
conversion analogique num�rique lorsque une demande de
mesure est effectu�e

Les fonctions associ�es sont~:
\begin{description}
\item[signed int Lire\_Vitesse()] renvoie un entier sign�
  entre -512 et +511 proportionnel � la vitesse.
\item[float Lire\_Vitesse\_float()] renvoie un flottant
  entre -3.3 et +3.3 qui repr�sente la vitesse en
  m.s\up{-1}.
\end{description}
       
\subsection{Commande du courant -- PWM --}
La consigne de courant envoy�e au r�gulateur de courant est
un signal PWM.

Les fonctions associ�es sont~:
\begin{description}
\item[void Fixe\_Rapport(float Commande\_Courant)] fixe le
  rapport cyclique de l'unit� PWM � la valeur\\
  \texttt{Commande\_Courant} au format flottant $\in
  [-1.0;+1.0]$~; l'annulation du courant peut �tre effectu�e
  par l'appel de la fonction \texttt{Fixe\_Rapport(0.0)}.
\end{description}


