\documentclass[xcolor=dvipsnames,onecolumn,french]{IEEEtran}
\usepackage[french]{babel}
\usepackage[latin1]{inputenc} 
\usepackage{times}
\usepackage[T1]{fontenc}
\usepackage{verbatim}
\usepackage{graphicx}
\usepackage{color}
\usepackage[dvipsnames]{xcolor}
%\usepackage{fix2col}

\usepackage{expdlist}
\makeatletter

%\usepackage{listings}
\usepackage{../../lst_asm}
\makeatother

\usepackage{babel}
\begin{document}

\title{TD1  Asm Cortex-M3 }


\author{Pascal Acco pour l'�quipe des branquignolles}

\maketitle
\begin{abstract}
	Apr�s une petite mise en bouche sur les sections de donn�es et la taille des transferts. Vous traduirez un algorithme �crit en langage C pour le stm32 utilisant le GPIO. Cet algorithme peut �tre tr�s facilement adapt� au c\oe{}ur con�u en SFO : vous aurez ainsi un exemple de programme en langage d'assemblage dont les instructions (ou leur �quivalentes) formeront le jeu d'instruction minimal de votre c\oe{}ur.
\end{abstract}


\section{Petites questions rapides}
Dans le module dummy.s figurent les d�clarations et le code suivant :

\begin{lstlisting}

CAKE   equ 10
;_______________________________________________________________________
	AREA    DummyVars,    DATA,    READWRITE
Caracteres	DCB		12,    -1,    'A'
Nombre		DCW		-2,    0xACDC,    ((CAKE<<12) + (0xC<<8) + (12<<0x4))
Miam_Un		SPACE	CAKE
Fin_Cake

;_______________________________________________________________________
	AREA 	DummyCode,    CODE,    READONLY

main PROC

	MOV		R2,  #CAKE
	LDR		R1,  =Fin_Cake
	LDR		R0,  =DummyVars
____Repete
		LDRH 	R7,  [R0]
		LDRSH 	R8,  [R0]
		ADD		R0,  #2
	B ____Repete
	
 ENDP ; fin du main
 
\end{lstlisting}

L'�diteur de lien d�cide que \texttt{dummyVars} vaut 0x20000000. Indiquez ce que vallent les registres R0, R7, R8 � chaque passage sur l'instruction \texttt{B Repete} . Que vallent R2 et R1 ?

% 0000FF0C			FFFFFF0C
% 00000041			0000041
% 0000FFFE			FFFFFFFE
% 0000ACDC		FFFFACDC
% 0000ACC0 		FFFFACC0



\section{Probl�me : Jeu de r�flexe }
Le programme suivant utilise le GPIO du STM32F103RB, puce incluant le cortex-M3 et des p�riph�riques, qui est similaire dans son fonctionnement � celui que vous avez con�us en SFO. La seule diff�rence est que le CR utilise 4 bits pour configurer une seule broche du port car la configuration est plus fine qu'un simple input ou output. Le jeu de r�flexe utilise deux boutons poussoirs (PORT.0 et PORT.3) et deux diodes (PORT.1 et PORT2). Les joueurs attendent � l'affut que les diodes se mettent � clignoter et appuient le plus rapidement possible sur leur bouton poussoir. D�s ce moment seule la diode du joueur le plus rapide reste allum�e pendant un certain temps pour indiquer le gagnant. Les diodes s'�teignent ensuite et les joueur retournent aux aguets...

\pagebreak

\begin{lstlisting}[language={[ANSI]C}, frameround=tttt,frame=single,extendedchars=true]
#define IN 0x8 // input with pulldown
#define OUT 0x2 // output 2MHz with push/pull

unsigned int * Cr = (unsigned int *)  0x40010C00; // CRL of GPIOB (STM32f10x) 
unsigned int * Idr = (unsigned int *) 0x40010C08; // ODR
unsigned int * Odr = (unsigned int *) 0x40010C0C; // IDR

unsigned char Boutons_Lus;

void Wait(int);

int main(void)
{   unsigned char Pas_Encore ; 
	unsigned int mask;

	// configure |PORT.3 	  |PORT.2 	    |PORT.1 	  |PORT.0
	//    as     |In	  	  |Out    	    |Out    	  |In
	//___________|____________|_____________|_____________|___________
	//  for      |Player1  	  |Player1  	|Player 2     |Player 2
	//           |Button      |Led 		    |Led		  |Button
	mask = 	      (IN<<(3*4)) +(OUT<<(2*4)) +(OUT<<(1*4)) +(IN<<(0*4));
	*Cr = mask;

	while(1)
	{   Pas_Encore = 'R'; // "Run" car les joueurs n'ont pas encore appuy� sur une touche

		// Fait clignoter les LEDs en attendants que les players appuient
		while (Pas_Encore)  // rappel : 0=> false ; tout le reste (et donc 'R')  => true 
		{	// Teste si au moins un bouton est appuy� : on teste donc si IDR.0 (bouton player 1) 
			//    ou IDR.3 (bouton player 2 ) est � 1, gr�ce au masque binaire 2_00001001 = 0x9
			Boutons_Lus = *Idr & 0x9 ;  // mets tous les bits � 0 sauf les bits .0 et .3
			
			// Boutons_Lus vaut 0 (!False) si aucun bouton appuy�  et true sinon  
			if (Boutons_Lus) Pas_Encore =0;

			// Change �tat des diodes avec un XOR bit � bit du masque 2_00000110
			*Odr ^= 0x6   ;
		}//while pas appuy�

		//etteinds les diodes
		*Odr &=  ~(6); //~  est l'inversion de chaque bits

		// affiche le vainqueur ou ex-aequo
		if (Boutons_Lus & (1<<0)) *Odr |= (1<<1);
		if (Boutons_Lus & (1<<3)) *Odr |= (1<<2);
		  
		// attends qu'ils aient vu le resultat
		Wait(1000000) ;

		//etteinds les diodes
		*Odr &=  ~(6); //~  est l'inversion de chaque bits

		// les joueurs attendent le signal pour appuyer
		Wait(3000000) ;
	}// while 1
} // main

// Boucle de temporisation 
void Wait( int N)
{ 	volatile int cpt ;
	
	cpt=0;
	for (cpt=0;cpt<N;cpt++);
}


\end{lstlisting}
Traduisez cet algorithme en assembleur en indiquant l'association entre registres et variables du programme. 

\end{document}
